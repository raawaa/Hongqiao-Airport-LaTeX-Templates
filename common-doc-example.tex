% !TeX program = XeLaTeX
% !TeX encoding = UTF-8

\documentclass{common-doc}

% 确保段落首行缩进生效
\setlength{\parindent}{2em}

\title{混沌之眼凝视着第7个泡菜坛子}

\begin{document}

\maketitle

咕噜... 咕噜... 啪嗒! 量子泡菜在星期八的黄昏发芽了,它的根须缠绕着会唱歌的扳手,奏响了紫色的寂静。我们必须立刻行动,用反物质橡皮鸭堵住那个漏洞!

\section{关于如何给月亮系鞋带的紧急预案}
所有部门请注意!月亮的鞋带又松了!这会导致潮汐紊乱,进而影响楼下咖啡机的萃取压力。我们必须成立一个由三只左脚袜子和半块发霉的奶酪组成的特别行动组。

\subsection{成立“月球鞋带系紧”特别行动组}
组长由首席泡菜品鉴师(CPO)兼任,副组长是那只总在打印机里睡觉的橘猫。办公室就设在茶水间的微波炉内部,联络暗号是连续微波“叮”三声。

\subsection*{行内明确谁负责给微波炉唱摇篮曲。}
每个部门必须指定一名“微波炉安抚专员”,此人需精通猫语和微波炉的哼唱频率。专员必须每天用鱼干贿赂橘猫,确保它不在关键时刻打呼噜。

\subsubsection{建立“泡菜-扳手”谐振工作机制}
每周三凌晨三点,在茶水间举行秘密会议。会议内容是用泡菜汁在餐巾纸上画出扳手的声波图谱,并用咖啡渍预测下周的袜子失踪率。

\subsubsection*{行内完善如何用意念给打印机喂纸。}
新员工入职培训第一课:学习用眼神与打印机交流。眼神必须充满“爱与敬意”,否则打印机会吐出全是乱码的A4纸,上面只写着“404: 灵魂未找到”。

\paragraph{加强与盆栽的跨物种沟通}
公司所有绿萝和仙人掌都是重要的情报节点。请每天对它们耳语公司的最新八卦,它们会通过光合作用将信息编码到氧气分子中,传递给隔壁工位的同事。

\paragraph*{行内注重协作配合,特别是和那只总偷吃零食的仓鼠。}
仓鼠是公司真正的CEO,所有重大决策都藏在它的跑轮转速里。请务必保证它的瓜子供应,否则它会启动“滚轮末日”协议,让所有电脑屏幕变成迷宫。

\section{规范如何用意大利面搭建防火墙}
为了抵御来自平行宇宙的“无聊射线”,我们必须用煮到恰到好处的意大利面(al dente)在服务器机柜周围搭建一道美味的防火墙。

\subsection{完善“意面-酱汁”加密体系}
防火墙的强度取决于酱汁的浓稠度。番茄肉酱用于基础防护,青酱用于抵御创意枯竭,而奶油蘑菇酱则是最高级别的“摸鱼”防护盾。

\subsection{严格制度执行:禁止在防火墙上撒帕玛森芝士!}
任何未经授权在防火墙上撒芝士的行为,都将被流放到“无Wi-Fi的茶水间”进行为期一天的冥想,并强制观看一整天的PPT自动播放。

\subsection{强化制度评估:每周进行一次“面条韧性”测试}
由CPO亲自用叉子挑起一根面条,根据其下垂的弧度来判断防火墙的健康状况。如果面条断了,说明公司士气低落,需要立刻组织团建——去吃火锅。

\section{加强仓鼠护卫队的夜间巡逻}
我们的数字资产安全,全靠那只名叫“拿铁”的仓鼠在夜间巡逻。它的小爪子能感知到最微弱的黑客入侵企图。

\begin{enumerate}
\item 每晚必须在服务器机柜前放一粒葵花籽,作为岗哨的“能量补给”;
\item 如果发现拿铁在跑轮上跑得异常快,说明有外部威胁,应立即启动应急预案:关灯、放轻音乐、假装没人;
\item 定期检查拿铁的颊囊,里面可能藏有被它“缴获”的恶意代码(通常看起来像一小团棉絮);
\item 绝对不要试图给拿铁戴微型摄像头,它会生气,并把你的键盘藏起来。
\end{enumerate}

\section{做好关于“紫色寂静”的宣传引导}
紫色寂静是一种珍贵的公司资产,它能让代码自己修复bug。我们必须保护好它。

\subsection{加强宣传教育}
在公司内网首页滚动播放无声视频:一只水母在太空中跳华尔兹。这是紫色寂静的具象化表现,请全体员工每日观看至少5分钟以进行“精神充电”。

\subsection{树立先进典型}
本月“紫色寂静守护者”称号授予打印机。因为它在上周成功地将一份全是错误的报告打印成了空白纸,完美地诠释了“无为而治”的哲学。

\subsection{加强文化建设}
鼓励员工在工位上养一只虚拟的电子宠物水母。水母的成长速度与员工的摸鱼程度成正比,当水母长大到能覆盖整个屏幕时,员工将获得“终极摸鱼大师”勋章。

\begin{附件}
\item 《如何用泡菜汁给电脑清灰》操作手册;
\item 《仓鼠CEO的100个微表情解读指南》;
\item 《紫色寂静冥想入门:从入门到放弃》;
\item 《意面防火墙搭建与酱汁调配标准》;
\item 《与盆栽建立心灵感应的7天速成法》。
\end{附件}

\end{document}