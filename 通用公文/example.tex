% !TeX program = XeLaTeX
% !TeX encoding = UTF-8

\documentclass{common-doc}

% 确保段落首行缩进生效
\setlength{\parindent}{2em}

\title{关于加强机场安全管理工作的通知}

\begin{document}

\maketitle

根据《中华人民共和国民用航空安全管理规定》和上级主管部门的要求,为进一步加强机场安全管理工作,确保机场运行安全,现就有关事项通知如下:

\section{第一级提高安全意识,落实安全责任}
各部门、各子公司要充分认识安全工作的重要性,切实提高安全意识,严格落实安全主体责任。部门负责人要亲自抓安全,建立健全安全管理制度,确保安全工作层层落实到位。

\subsection{第二级加强组织领导}
成立由主要领导任组长的安全管理领导小组,定期召开安全工作会议,研究解决安全工作中的重大问题。

\subsection*{第二级行内完善安全制度。}
要根据实际情况,及时修订和完善安全管理制度,确保制度的科学性、有效性和可操作性。

\subsubsection{第三级具体措施}
建立健全各项安全管理制度,明确各岗位安全职责。

\subsubsection*{第三级行内责任分工。}
明确各部门、各岗位的安全管理职责,形成团结一致的工作格局。

\paragraph{第四级具体要求}
各部门、各岗位要严格遵守安全管理制度,不得违反安全职责。

\paragraph*{第四级行内要求}
各部门、各岗位要严格遵守安全管理制度,不得违反安全职责。

\section{强化安全检查,消除安全隐患}
各部门、各子公司要定期开展安全检查,及时发现和消除安全隐患。

\begin{enumerate}
\item 每月至少组织一次全面的安全检查;
\item 重点检查消防设施、应急设备、通信系统等关键设施设备;
\item 对检查中发现的问题,要立即整改,确保问题及时解决;
\item 建立安全检查台账,如实记录检查情况和整改结果。
\end{enumerate}

\section{加强应急演练,提高应急处置能力}
各部门、各子公司要加强应急管理,制定完善应急预案,定期组织应急演练。

\subsection{完善应急预案}
要根据机场运行特点和可能发生的突发事件,制定科学、合理、可操作的应急预案。

\subsection{开展应急演练}
每季度至少组织一次应急演练,通过演练检验应急预案的有效性,提高应急处置能力。

\subsection{加强应急队伍建设}
要组建专业的应急队伍,加强培训和管理,确保在突发事件发生时能够快速响应、有效处置。

\section{加强安全培训,提高员工安全素质}
各部门、各子公司要加强员工安全培训,提高员工安全意识和操作技能。

\subsection{制定培训计划}
要根据员工岗位特点和安全工作需要,制定详细的安全培训计划。

\subsection{开展培训活动}
定期组织员工参加安全培训,培训内容包括安全法律法规、安全操作规程、应急处置等方面。

\subsection{考核培训效果}
要对培训效果进行考核,确保员工真正掌握安全知识和技能。


\end{document}